\documentclass[11pt, oneside]{article}      % use "amsart" instead of "article" for AMSLaTeX format
\usepackage{geometry}                       % See geometry.pdf to learn the layout options. There are lots.
\geometry{letterpaper}                          % ... or a4paper or a5paper or ... 
% \geometry{landscape}                       % Activate for rotated page geometry
\usepackage[parfill]{parskip}          	% Activate to begin paragraphs with an empty line rather than an indent
% \usepackage{graphicx}
\usepackage{amsmath,amsthm,amssymb}
\usepackage{mathtools}
\usepackage{xcolor}
\usepackage[font=small,labelfont=bf,skip=0pt,justification=justified]{caption}
\usepackage{stackengine}
\usepackage{natbib}
\usepackage{stmaryrd}
\usepackage[most]{tcolorbox}
\usepackage[export]{adjustbox}
\usepackage{tikzit}
\input{zx.tikzdefs}
\input{zx.tikzstyles}

\usepackage{hyperref} % must be last package used

\bibliographystyle{abbrvnat}
\setcitestyle{square, authoryear}

\allowdisplaybreaks
\setlength{\jot}{20pt}

\theoremstyle{definition}
\newtheorem{theorem}{Theorem}[section]
\newtheorem{corollary}[theorem]{Corollary}
\newtheorem{lemma}[theorem]{Lemma}
\newtheorem*{lemma*}{Lemma}
\newtheorem*{proposition*}{Proposition}
\newtheorem{proposition}[theorem]{Proposition}
\newtheorem{conjecture}[theorem]{Conjecture}
\newtheorem{definition}[theorem]{Definition}
\newtheorem{example}[theorem]{Example}
\newtheorem{remark}[theorem]{Remark}
\newtheorem{warning}[theorem]{Warning}

\title{\textbf{Qudit ZX-Diagram Simplification (Draft)}}
\author{
  Alex Townsend-Teague
  \and
  Quanlong Wang
  \and
  Konstantinos Meichantzidis
}
\date{}

\begin{document}
\maketitle
\begin{abstract}
\end{abstract}

\section{Introduction}

Diagrams are read bottom to top. `It is not hard to believe that qubits and qutrits behave differently from qupits.' \citep{Cui_2015}. Can't decide on notation for Hadamard stuff. Numbered yellow boxes are good for graph-like diagrams, but red/green split boxes are great for colour change rules:

\begin{equation}
	\tikzfig{rules/colour_change/split/lhs} \quad = \quad 
	\tikzfig{rules/colour_change/rhs}
	\hspace{60pt}
	\tikzfig{rules/colour_change/split/flip_lhs} \quad = \quad 
	\tikzfig{rules/colour_change/flip_rhs}
\end{equation}

And then the dualiser has a natural alternative notation - important because using a yellow box with a $D$ on it now looks more like a Hadamard with decoration $D$:

\begin{equation}
	\tikzfig{rules/snake/split/hh} \quad = \quad 
	\tikzfig{rules/snake/split/-h-h} \quad = \quad
	\tikzfig{rules/snake/split/dualiser_red} \quad = \quad
	\tikzfig{rules/snake/split/dualiser_green}
\end{equation}

But the split boxes have the drawback that often you can no longer say `the equation holds with the roles of green and red interchanged' - e.g. the Euler decomposition of the Hadamard. Not a problem for dualiser though. 

Qudit rules are shown in Figure \ref{fig:qudit_rules}. We're mostly interested in stabilizer stuff so will give phase components as multiples of $\frac{2\pi}{d}$.
\begin{itemize}
	\item $\vec{\kappa} = (1, 2, ..., d-1)$
	\item $\vec{\tau}$ has components $\vec{\tau}_j = \frac{j}{2}(j+d)$
	\item $\vec{\alpha}^n$ has components $\vec{\alpha}^n_j = \vec{\alpha}_{j-n} - \vec{\alpha}_{-n}$
	\item Indices are mod $d$, and $\vec{\alpha}_0 = 0$ for any phase $a$.
	\item $\cev{\alpha} = (\vec{\alpha}_d-1, \vec{\alpha}_d-2, ...  , \vec{\alpha}_1)$
\end{itemize}

\begin{figure}
	\begin{tcolorbox}[colback=white]
		\begin{equation*}
			\tikzfig{rules/fusion} \quad \hypertarget{rule_fusion}{\mathbf{(f)}}
		\end{equation*}
		% \vspace{5pt}
		\begin{equation*}
			\tikzfig{rules/identity/lhs} \quad = \quad 
			\tikzfig{rules/identity/rhs} \quad \hypertarget{rule_id}{\mathbf{(id)}}
			\hspace{60pt}
			\tikzfig{rules/green/lhs} \quad = \quad 
			\tikzfig{rules/green/rhs} \quad \hypertarget{rule_green_cup}{\mathbf{(g)}}
		\end{equation*}
		\vspace{5pt}
		\begin{equation*}
			\tikzfig{rules/0_copy/lhs} \quad = \quad 
			\tikzfig{rules/0_copy/rhs} \quad \hypertarget{rule_0_copy}{\mathbf{(cp_0)}}
			\hspace{60pt}
			\tikzfig{rules/bialgebra/lhs} \quad = \quad 
			\tikzfig{rules/bialgebra/rhs} \quad \hypertarget{rule_bialgebra}{\mathbf{(b)}}
		\end{equation*}
		\vspace{5pt}
		\begin{equation*}
			\tikzfig{rules/k_copy/lhs} \quad = \quad 
			\tikzfig{rules/k_copy/rhs} \quad \hypertarget{rule_k_copy}{\mathbf{(cp_{\kappa})}}
			\hspace{60pt}
			\tikzfig{rules/commute/lhs} \quad = \quad 
			\tikzfig{rules/commute/rhs} \quad \hypertarget{rule_commute}{\mathbf{(cm)}}
		\end{equation*}
		% \vspace{5pt}
		\begin{equation*}
			\tikzfig{rules/hadamard/h_hdagger} \quad = \quad 
			\tikzfig{rules/hadamard/identity} \quad = \quad 
			\tikzfig{rules/hadamard/hdagger_h} \quad \hypertarget{rule_hadamard}{\mathbf{(H)}}
			\hspace{60pt}
			\tikzfig{rules/hadamard/h} \quad = \quad 
			\tikzfig{rules/hadamard/decomposition} \quad \hypertarget{rule_euler}{\mathbf{(E)}}
		\end{equation*}
		\vspace{5pt}
		\begin{equation*}
			\tikzfig{rules/colour_change/lhs} \quad = \quad 
			\tikzfig{rules/colour_change/rhs} \quad \hypertarget{rule_colour_change}{\mathbf{(cc)}}
			\hspace{60pt}
			\tikzfig{rules/colour_change/flip_lhs} \quad = \quad 
			\tikzfig{rules/colour_change/flip_rhs} \quad \mathbf{(cc)}
		\end{equation*}
		\vspace{5pt}
		\begin{equation*}
			\tikzfig{rules/snake/snake} \quad = \quad 
			\tikzfig{rules/snake/multi_wire} \quad = \quad 
			\tikzfig{rules/snake/hadamards} \quad = \quad 
			\tikzfig{rules/snake/hadamard_adjoints} \quad = \quad
			\tikzfig{rules/snake/split/dualiser_red} \quad = \quad
			\tikzfig{rules/snake/split/dualiser_green} \quad \hypertarget{rule_snake}{\mathbf{(s)}}
		\end{equation*}
	\end{tcolorbox}
	\vspace{5pt}
	\caption{Qudit ZX-calculus rules}
	\label{fig:qudit_rules}

\end{figure}

\section{Stabilizer Phases}

We want to know what spider phases we might encounter in the world of stabilizer qudit ZX-calculus. For odd $d$, components $\vec{a}_j$ of stabilizer phases $\vec{a}$ will be integer multiples of $\frac{2\pi}{d}$. For even $d$, stabilizer components $\vec{a}_j$ can be half-integer multiples of $\frac{2\pi}{d}$ (i.e. integer multiples of $\frac{\pi}{d}$). The set of stabilizer phases is a subset of the set of all possible $(d-1)$-tuples of such integers/half-integers. In general, this subset is strict, though we note that for qubits and qutrits it is not.

The shift gate $S$ is uniquely defined by its action on the Pauli gates $X$ and $Z$. Specifically, it satisfies:

\begin{equation}
	SXS^\dagger = XZ \quad,\quad SZS^\dagger = Z
\end{equation}

In ZX-parlance, these equations are:

\begin{equation}
	\tikzfig{clifford/shift/X/1} =~~ \tikzfig{clifford/shift/X/2} \quad,
	\hspace{30pt}
	\tikzfig{clifford/shift/Z/1} =~~ \tikzfig{clifford/shift/Z/2}
\end{equation}

These are satisfied by the following $Z$-gate whose phase $\vec{s}$ has components $\vec{s}_j$:

\begin{equation}
	\tikzfig{clifford/shift/def/1} = \tikzfig{clifford/shift/def/2} \quad,\quad 
	\vec{s}_j = \frac{j}{2}(j + 2 - d)
\end{equation}

Note that for even $d$, the components can be half-integers - for example, $\vec{s}_{d-1} = \frac{d-1}{2}$. For all $d$, the qudit Clifford group on $n$ qubits is generated by the Hadamard gate $H$, the shift gate $S$ and the generalised $CNOT$ gate \citep{Farinholt_2014}:

\begin{equation}
	\tikzfig{clifford/gens/H} \quad,\hspace{30pt}
	\tikzfig{clifford/gens/S} \quad,\hspace{30pt}
	\tikzfig{clifford/gens/CNOT}
\end{equation}

The Clifford group on one qudit (the \emph{local Clifford group}) is generated by $H$ and $S$ \citep{Farinholt_2014}. We seek a normal form for such gates, \`{a} la \citep{Wang_2018}{Theorem 6.2.12}. \textbf{So far this has proved difficult!} I've been playing around a bit with gates with the phase $\p{k}{x}$, whose components are:

\begin{equation}
	 \p{k}{x}_j = -j\left(\frac{j+1}{2}k + x\right)
\end{equation} 

These are analogous to those in \cite{Wang_2018}{Lemma 6.2.8} and have the following properties:

\begin{enumerate}
	\item $\tikzfig{clifford/p/def/1} = \tikzfig{clifford/p/def/2}$
	\item They form an Abelian subgroup: 
	\begin{enumerate}
		\item $\p{k}{x} + \p{l}{y} = \p{k+l}{x+y}$
		\item $-\p{k}{x} = \p{-k}{-x}$
		\item identity is $\p{0}{0} = (0, ..., 0)$
	\end{enumerate}
	\item They include a lot of our favourite phases:
	\begin{enumerate}
	 	\item $\vec{\kappa} = \p{0}{-1}$
	 	\item $\vec{\tau} = \p{-1}{\frac{1-d}{2}}$
	 	\item $\vec{s} = \p{-1}{\frac{d-1}{2}}$
	 	\item $\cev{s} = \p{-1}{\frac{d+3}{2}}$
	 \end{enumerate}
	\item The last component $\p{k}{x}_{d-1}$ is exactly $x$.
\end{enumerate}

\section{Local Complementation and Pivot}

Local complementation carries over to qudit case without much fuss. Some lemmas we'll need:

\begin{lemma}
	\begin{equation}
		\tikzfig{cup/1} ~~=~~ \tikzfig{cup/8} \quad,\hspace{30pt}
		\tikzfig{tau/1} ~~=~~ \tikzfig{tau/5} \quad,\hspace{30pt}
		\tikzfig{triangle/dualiser_lemma/1} ~~=~~ \tikzfig{triangle/dualiser_lemma/6}
	\end{equation}
	\begin{proof}
		For the first equation:
		\begin{equation}	
			\begin{split}
				\tikzfig{cup/1} ~~=~~ 
				\tikzfig{cup/2} ~~=~~ 
				\tikzfig{cup/3} ~~=~~ 
				\tikzfig{cup/4}\\ ~~=~~
				\tikzfig{cup/5} ~~=~~ 
				\tikzfig{cup/6} ~~=~~ 
				\tikzfig{cup/7} ~~=~~ 
				\tikzfig{cup/8}
			\end{split}
		\end{equation}

		For the second:

		\begin{equation}
			\tikzfig{tau/1} ~~=~~ 
			\tikzfig{tau/2} ~~=~~ 
			\tikzfig{tau/3} ~~=~~ 
			\tikzfig{tau/4} ~~=~~ 
			\tikzfig{tau/5}
		\end{equation}

		And the third:

		\begin{equation}
			\tikzfig{triangle/dualiser_lemma/1} ~~=~~ 
			\tikzfig{triangle/dualiser_lemma/2} ~~=~~ 
			\tikzfig{triangle/dualiser_lemma/3} ~~=~~ 
			\tikzfig{triangle/dualiser_lemma/4} ~~=~~ 
			\tikzfig{triangle/dualiser_lemma/5} ~~=~~ 
			\tikzfig{triangle/dualiser_lemma/6}
		\end{equation}
	\end{proof}
\end{lemma}

So, first we have a triangle lemma:

\begin{lemma}
	\begin{equation}
		\tikzfig{triangle/basic_lemma/1} ~~=~~ \tikzfig{triangle/basic_lemma/11}
	\end{equation}
	\begin{proof}
		\begin{align*}
			&~~~~~ \tikzfig{triangle/basic_lemma/1} &&~~=~~ \tikzfig{triangle/basic_lemma/2} &&&~~=~~ \tikzfig{triangle/basic_lemma/3} \\
			&~~=~~ \tikzfig{triangle/basic_lemma/4} &&~~=~~ \tikzfig{triangle/basic_lemma/5} &&&~~=~~ \tikzfig{triangle/basic_lemma/6} \\
			&~~=~~ \tikzfig{triangle/basic_lemma/7} &&~~=~~ \tikzfig{triangle/basic_lemma/8} &&&~~=~~ \tikzfig{triangle/basic_lemma/9} \\
			&~~=~~ \tikzfig{triangle/basic_lemma/10} &&~~=~~ \tikzfig{triangle/basic_lemma/11}
		\end{align*}
		
	\end{proof}
\end{lemma}

Then a lemma that tells us how to remove one edge in the $K_3$ graph state:

\begin{lemma}
	\begin{equation}
		\tikzfig{triangle/graph_state_lemma/1} ~~=~~ \tikzfig{triangle/graph_state_lemma/13}
	\end{equation}
	\begin{proof}
		\begin{align*}
			&~~~~~ \tikzfig{triangle/graph_state_lemma/2} &&~~=~~ \tikzfig{triangle/graph_state_lemma/3} &&&~~=~~ \tikzfig{triangle/graph_state_lemma/4} \\
			&~~=~~ \tikzfig{triangle/graph_state_lemma/5} &&~~=~~ \tikzfig{triangle/graph_state_lemma/6} &&&~~=~~ \tikzfig{triangle/graph_state_lemma/7} \\
			&~~=~~ \tikzfig{triangle/graph_state_lemma/8} &&~~=~~ \tikzfig{triangle/graph_state_lemma/9} &&&~~=~~ \tikzfig{triangle/graph_state_lemma/10} \\
			&~~=~~ \tikzfig{triangle/graph_state_lemma/11} &&~~=~~ \tikzfig{triangle/graph_state_lemma/12}
		\end{align*}
		
	\end{proof}
\end{lemma}

And finally the local complementation result for the complete graph $K_3$:

\begin{lemma}
	\begin{equation}
		\tikzfig{triangle/graph_state/1} ~~=~~ \tikzfig{triangle/graph_state/8}
	\end{equation}
	\begin{proof}
		\begin{align*}
			&~~~~~ \tikzfig{triangle/graph_state/1} &&~~=~~ \tikzfig{triangle/graph_state/2} &&&~~=~~ \tikzfig{triangle/graph_state/3} \\
			&~~=~~ \tikzfig{triangle/graph_state/4} &&~~=~~ \tikzfig{triangle/graph_state/5} &&&~~=~~ \tikzfig{triangle/graph_state/6} \\
			&~~=~~ \tikzfig{triangle/graph_state/7}
		\end{align*}
		
	\end{proof}
\end{lemma}

Other base cases $K_0, K_1$ and $K_2$ are easy enough to prove. For larger complete graphs $K_n$ we proceed inductively:

\begin{lemma}
	\begin{equation}
		\tikzfig{local_comp/K_n/1} ~~=~~ \tikzfig{local_comp/K_n/16}
	\end{equation}
	\begin{proof}
		\begin{align*}
			&~~~~~ \tikzfig{local_comp/K_n/1} &&~~=~~ \tikzfig{local_comp/K_n/2} &&&~~=~~ \tikzfig{local_comp/K_n/3} \\
			&~~=~~ \tikzfig{local_comp/K_n/4} &&~~=~~ \tikzfig{local_comp/K_n/5} &&&~~=~~ \tikzfig{local_comp/K_n/6} \\
			&~~=~~ \tikzfig{local_comp/K_n/7} &&~~=~~ \tikzfig{local_comp/K_n/8} &&&~~=~~ \tikzfig{local_comp/K_n/9} \\
			&~~=~~ \tikzfig{local_comp/K_n/10} &&~~=~~ \tikzfig{local_comp/K_n/11} &&&~~=~~ \tikzfig{local_comp/K_n/12} \\
			&~~=~~ \tikzfig{local_comp/K_n/13} &&~~=~~ \tikzfig{local_comp/K_n/14} &&&~~=~~ \tikzfig{local_comp/K_n/15} \\
			&~~=~~ \tikzfig{local_comp/K_n/16}
		\end{align*}
		
	\end{proof}
\end{lemma}

For general (non-complete) graphs, can derive the result from the $K_n$ case. Then need to derive pivot equation. 

\section{Elimination Theorems}

Find as many eliminatable spiders as possible!

\bibliography{qudit_simplification}

\appendix

\end{document} 